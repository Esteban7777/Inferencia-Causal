% Options for packages loaded elsewhere
\PassOptionsToPackage{unicode}{hyperref}
\PassOptionsToPackage{hyphens}{url}
%
\documentclass[
]{article}
\usepackage{amsmath,amssymb}
\usepackage{iftex}
\ifPDFTeX
  \usepackage[T1]{fontenc}
  \usepackage[utf8]{inputenc}
  \usepackage{textcomp} % provide euro and other symbols
\else % if luatex or xetex
  \usepackage{unicode-math} % this also loads fontspec
  \defaultfontfeatures{Scale=MatchLowercase}
  \defaultfontfeatures[\rmfamily]{Ligatures=TeX,Scale=1}
\fi
\usepackage{lmodern}
\ifPDFTeX\else
  % xetex/luatex font selection
\fi
% Use upquote if available, for straight quotes in verbatim environments
\IfFileExists{upquote.sty}{\usepackage{upquote}}{}
\IfFileExists{microtype.sty}{% use microtype if available
  \usepackage[]{microtype}
  \UseMicrotypeSet[protrusion]{basicmath} % disable protrusion for tt fonts
}{}
\makeatletter
\@ifundefined{KOMAClassName}{% if non-KOMA class
  \IfFileExists{parskip.sty}{%
    \usepackage{parskip}
  }{% else
    \setlength{\parindent}{0pt}
    \setlength{\parskip}{6pt plus 2pt minus 1pt}}
}{% if KOMA class
  \KOMAoptions{parskip=half}}
\makeatother
\usepackage{xcolor}
\usepackage[margin=1in]{geometry}
\usepackage{color}
\usepackage{fancyvrb}
\newcommand{\VerbBar}{|}
\newcommand{\VERB}{\Verb[commandchars=\\\{\}]}
\DefineVerbatimEnvironment{Highlighting}{Verbatim}{commandchars=\\\{\}}
% Add ',fontsize=\small' for more characters per line
\usepackage{framed}
\definecolor{shadecolor}{RGB}{248,248,248}
\newenvironment{Shaded}{\begin{snugshade}}{\end{snugshade}}
\newcommand{\AlertTok}[1]{\textcolor[rgb]{0.94,0.16,0.16}{#1}}
\newcommand{\AnnotationTok}[1]{\textcolor[rgb]{0.56,0.35,0.01}{\textbf{\textit{#1}}}}
\newcommand{\AttributeTok}[1]{\textcolor[rgb]{0.13,0.29,0.53}{#1}}
\newcommand{\BaseNTok}[1]{\textcolor[rgb]{0.00,0.00,0.81}{#1}}
\newcommand{\BuiltInTok}[1]{#1}
\newcommand{\CharTok}[1]{\textcolor[rgb]{0.31,0.60,0.02}{#1}}
\newcommand{\CommentTok}[1]{\textcolor[rgb]{0.56,0.35,0.01}{\textit{#1}}}
\newcommand{\CommentVarTok}[1]{\textcolor[rgb]{0.56,0.35,0.01}{\textbf{\textit{#1}}}}
\newcommand{\ConstantTok}[1]{\textcolor[rgb]{0.56,0.35,0.01}{#1}}
\newcommand{\ControlFlowTok}[1]{\textcolor[rgb]{0.13,0.29,0.53}{\textbf{#1}}}
\newcommand{\DataTypeTok}[1]{\textcolor[rgb]{0.13,0.29,0.53}{#1}}
\newcommand{\DecValTok}[1]{\textcolor[rgb]{0.00,0.00,0.81}{#1}}
\newcommand{\DocumentationTok}[1]{\textcolor[rgb]{0.56,0.35,0.01}{\textbf{\textit{#1}}}}
\newcommand{\ErrorTok}[1]{\textcolor[rgb]{0.64,0.00,0.00}{\textbf{#1}}}
\newcommand{\ExtensionTok}[1]{#1}
\newcommand{\FloatTok}[1]{\textcolor[rgb]{0.00,0.00,0.81}{#1}}
\newcommand{\FunctionTok}[1]{\textcolor[rgb]{0.13,0.29,0.53}{\textbf{#1}}}
\newcommand{\ImportTok}[1]{#1}
\newcommand{\InformationTok}[1]{\textcolor[rgb]{0.56,0.35,0.01}{\textbf{\textit{#1}}}}
\newcommand{\KeywordTok}[1]{\textcolor[rgb]{0.13,0.29,0.53}{\textbf{#1}}}
\newcommand{\NormalTok}[1]{#1}
\newcommand{\OperatorTok}[1]{\textcolor[rgb]{0.81,0.36,0.00}{\textbf{#1}}}
\newcommand{\OtherTok}[1]{\textcolor[rgb]{0.56,0.35,0.01}{#1}}
\newcommand{\PreprocessorTok}[1]{\textcolor[rgb]{0.56,0.35,0.01}{\textit{#1}}}
\newcommand{\RegionMarkerTok}[1]{#1}
\newcommand{\SpecialCharTok}[1]{\textcolor[rgb]{0.81,0.36,0.00}{\textbf{#1}}}
\newcommand{\SpecialStringTok}[1]{\textcolor[rgb]{0.31,0.60,0.02}{#1}}
\newcommand{\StringTok}[1]{\textcolor[rgb]{0.31,0.60,0.02}{#1}}
\newcommand{\VariableTok}[1]{\textcolor[rgb]{0.00,0.00,0.00}{#1}}
\newcommand{\VerbatimStringTok}[1]{\textcolor[rgb]{0.31,0.60,0.02}{#1}}
\newcommand{\WarningTok}[1]{\textcolor[rgb]{0.56,0.35,0.01}{\textbf{\textit{#1}}}}
\usepackage{graphicx}
\makeatletter
\def\maxwidth{\ifdim\Gin@nat@width>\linewidth\linewidth\else\Gin@nat@width\fi}
\def\maxheight{\ifdim\Gin@nat@height>\textheight\textheight\else\Gin@nat@height\fi}
\makeatother
% Scale images if necessary, so that they will not overflow the page
% margins by default, and it is still possible to overwrite the defaults
% using explicit options in \includegraphics[width, height, ...]{}
\setkeys{Gin}{width=\maxwidth,height=\maxheight,keepaspectratio}
% Set default figure placement to htbp
\makeatletter
\def\fps@figure{htbp}
\makeatother
\setlength{\emergencystretch}{3em} % prevent overfull lines
\providecommand{\tightlist}{%
  \setlength{\itemsep}{0pt}\setlength{\parskip}{0pt}}
\setcounter{secnumdepth}{-\maxdimen} % remove section numbering
\ifLuaTeX
  \usepackage{selnolig}  % disable illegal ligatures
\fi
\usepackage{bookmark}
\IfFileExists{xurl.sty}{\usepackage{xurl}}{} % add URL line breaks if available
\urlstyle{same}
\hypersetup{
  pdftitle={Tarea 1 Simulación del efecto de la educación en los ingresos},
  pdfauthor={Wilmer Rojas \& John Esteban Londoño \& William Aguirre},
  hidelinks,
  pdfcreator={LaTeX via pandoc}}

\title{Tarea 1 Simulación del efecto de la educación en los ingresos}
\author{Wilmer Rojas \& John Esteban Londoño \& William Aguirre}
\date{2024-08-30}

\begin{document}
\maketitle

\section{Simulación}\label{simulaciuxf3n}

\subsection{Introducción}\label{introducciuxf3n}

El objetivo de este documento es presentar la simulación de la relación
causal de la educación sobre los ingresos de los individuos. Para ello
partimos del siguiente proceso generador de datos:
\[Ingreso= \beta\ +\alpha\ educación + \epsilon\]

Donde \(\alpha\) es el efecto que tiene la educación sobre el ingreso.
Para efectos de la simulación, asumimos que el tamaño de este efecto es
igual a 10. Adicionalmente, partimos del supuesto de que \(\epsilon\) no
está relacionado con la educación y por esta razón existe exogeneidad de
la variable de interés.

Para evidenciar la diferencia en la estimación por el método de mínimos
cuadrados ordinarios cuando existe exogeneidad y cuando se presenta una
correlación entre la variable explicativa (educación) y los no
observables (\(\epsilon\)) se genera una segunda simulación con el
siguiente procesos generador de datos:
\[Ingreso= \beta\ +\alpha\ educación + \epsilon\] \(Donde\)
\[E[educacion|\epsilon] \neq 0\]

Al comparar las estimaciones por M.C.O. de los datos simulados con los
dos procesos se logra evidenciar que el estimador es insesgado cuando se
cumple el principio de exogeneidad. Sin embargo, la estimación cuando se
viola este supuesto se aleja del parámetro poblacional debido al sesgo
de selección.

\begin{Shaded}
\begin{Highlighting}[]
\NormalTok{N}\OtherTok{=}\DecValTok{1000000}

\NormalTok{income\_aut }\OtherTok{=} \DecValTok{100}
\NormalTok{effect}\OtherTok{=}\DecValTok{10}

\NormalTok{df }\OtherTok{\textless{}{-}} \FunctionTok{tibble}\NormalTok{(}
    \AttributeTok{educ =} \FunctionTok{rnorm}\NormalTok{(N, }\AttributeTok{mean=}\DecValTok{11}\NormalTok{, }\AttributeTok{sd=} \DecValTok{9}\NormalTok{),}
    \AttributeTok{u =} \FunctionTok{rnorm}\NormalTok{(N, }\AttributeTok{mean=}\DecValTok{0}\NormalTok{, }\AttributeTok{sd=}\DecValTok{36}\NormalTok{),}
    \AttributeTok{income =}\NormalTok{ income\_aut }\SpecialCharTok{+}\NormalTok{ effect}\SpecialCharTok{*}\NormalTok{educ }\SpecialCharTok{+}\NormalTok{ u) }\SpecialCharTok{\%\textgreater{}\%} 
    \FunctionTok{mutate}\NormalTok{(}\AttributeTok{niv\_educ=}\FunctionTok{case\_when}\NormalTok{(educ }\SpecialCharTok{\textless{}=} \DecValTok{5} \SpecialCharTok{\textasciitilde{}} \DecValTok{0}\NormalTok{, }
\NormalTok{                              educ }\SpecialCharTok{\textgreater{}}\DecValTok{5} \SpecialCharTok{\&}\NormalTok{ educ }\SpecialCharTok{\textless{}=}\DecValTok{9} \SpecialCharTok{\textasciitilde{}} \DecValTok{5}\NormalTok{, }
\NormalTok{                              educ }\SpecialCharTok{\textgreater{}}\DecValTok{9} \SpecialCharTok{\&}\NormalTok{ educ }\SpecialCharTok{\textless{}=}\DecValTok{11} \SpecialCharTok{\textasciitilde{}} \DecValTok{9}\NormalTok{,}
\NormalTok{                              educ }\SpecialCharTok{\textgreater{}}\DecValTok{11} \SpecialCharTok{\&}\NormalTok{ educ }\SpecialCharTok{\textless{}=}\DecValTok{16} \SpecialCharTok{\textasciitilde{}} \DecValTok{11}\NormalTok{,}
\NormalTok{                              educ }\SpecialCharTok{\textgreater{}}\DecValTok{16} \SpecialCharTok{\&}\NormalTok{ educ }\SpecialCharTok{\textless{}=}\DecValTok{18} \SpecialCharTok{\textasciitilde{}} \DecValTok{16}\NormalTok{,}
\NormalTok{                              educ }\SpecialCharTok{\textgreater{}}\DecValTok{18} \SpecialCharTok{\&}\NormalTok{ educ }\SpecialCharTok{\textless{}=}\DecValTok{20} \SpecialCharTok{\textasciitilde{}} \DecValTok{18}\NormalTok{,}
\NormalTok{                              educ }\SpecialCharTok{\textgreater{}} \DecValTok{20} \SpecialCharTok{\textasciitilde{}} \DecValTok{20}\NormalTok{))}


\NormalTok{df\_1}\OtherTok{\textless{}{-}}\FunctionTok{rnorm\_multi}\NormalTok{(}\AttributeTok{n=}\DecValTok{1000000}\NormalTok{,}\AttributeTok{vars =} \DecValTok{2}\NormalTok{,}\AttributeTok{mu=}\FunctionTok{c}\NormalTok{(}\DecValTok{11}\NormalTok{,}\DecValTok{0}\NormalTok{),}\AttributeTok{sd=}\FunctionTok{c}\NormalTok{(}\DecValTok{9}\NormalTok{,}\DecValTok{36}\NormalTok{),}\AttributeTok{r=}\FloatTok{0.9}\NormalTok{,}\AttributeTok{varnames =} \FunctionTok{c}\NormalTok{(}\StringTok{\textquotesingle{}educ\textquotesingle{}}\NormalTok{,}\StringTok{\textquotesingle{}u\textquotesingle{}}\NormalTok{))}
\NormalTok{df\_i }\OtherTok{\textless{}{-}} \FunctionTok{tibble}\NormalTok{(}\AttributeTok{income=}\NormalTok{income\_aut }\SpecialCharTok{+}\NormalTok{ effect}\SpecialCharTok{*}\NormalTok{df\_1}\SpecialCharTok{$}\NormalTok{educ }\SpecialCharTok{+}\NormalTok{ df\_1}\SpecialCharTok{$}\NormalTok{u )}
\NormalTok{df\_1}\OtherTok{\textless{}{-}}\FunctionTok{cbind}\NormalTok{(df\_i,df\_1)}

\NormalTok{df\_1 }\OtherTok{\textless{}{-}}\NormalTok{ df\_1 }\SpecialCharTok{\%\textgreater{}\%}  \FunctionTok{mutate}\NormalTok{(}\AttributeTok{niv\_edu =} \FunctionTok{case\_when}\NormalTok{(educ }\SpecialCharTok{\textless{}=} \DecValTok{5} \SpecialCharTok{\textasciitilde{}} \DecValTok{0}\NormalTok{, }
\NormalTok{                              educ }\SpecialCharTok{\textgreater{}}\DecValTok{5} \SpecialCharTok{\&}\NormalTok{ educ }\SpecialCharTok{\textless{}=}\DecValTok{9} \SpecialCharTok{\textasciitilde{}} \DecValTok{5}\NormalTok{, }
\NormalTok{                              educ }\SpecialCharTok{\textgreater{}}\DecValTok{9} \SpecialCharTok{\&}\NormalTok{ educ }\SpecialCharTok{\textless{}=}\DecValTok{11} \SpecialCharTok{\textasciitilde{}} \DecValTok{9}\NormalTok{,}
\NormalTok{                              educ }\SpecialCharTok{\textgreater{}}\DecValTok{11} \SpecialCharTok{\&}\NormalTok{ educ }\SpecialCharTok{\textless{}=}\DecValTok{16} \SpecialCharTok{\textasciitilde{}} \DecValTok{11}\NormalTok{,}
\NormalTok{                              educ }\SpecialCharTok{\textgreater{}}\DecValTok{16} \SpecialCharTok{\&}\NormalTok{ educ }\SpecialCharTok{\textless{}=}\DecValTok{18} \SpecialCharTok{\textasciitilde{}} \DecValTok{16}\NormalTok{,}
\NormalTok{                              educ }\SpecialCharTok{\textgreater{}}\DecValTok{18} \SpecialCharTok{\&}\NormalTok{ educ }\SpecialCharTok{\textless{}=}\DecValTok{20} \SpecialCharTok{\textasciitilde{}} \DecValTok{18}\NormalTok{,}
\NormalTok{                              educ }\SpecialCharTok{\textgreater{}} \DecValTok{20} \SpecialCharTok{\textasciitilde{}} \DecValTok{20}\NormalTok{))}
\end{Highlighting}
\end{Shaded}

\begin{Shaded}
\begin{Highlighting}[]
\NormalTok{p1 }\OtherTok{\textless{}{-}} \FunctionTok{ggplot}\NormalTok{(df) }\SpecialCharTok{+} 
  \FunctionTok{geom\_histogram}\NormalTok{(}\FunctionTok{aes}\NormalTok{(income), }\AttributeTok{bins =} \DecValTok{100}\NormalTok{, }\AttributeTok{color =} \StringTok{"white"}\NormalTok{, }\AttributeTok{fill =} \StringTok{"green"}\NormalTok{, }\AttributeTok{alpha =} \FloatTok{0.5}\NormalTok{) }\SpecialCharTok{+} 
  \FunctionTok{geom\_histogram}\NormalTok{(}\FunctionTok{aes}\NormalTok{(educ), }\AttributeTok{bins =} \DecValTok{100}\NormalTok{, }\AttributeTok{color =} \StringTok{"white"}\NormalTok{, }\AttributeTok{fill =} \StringTok{"orange"}\NormalTok{, }\AttributeTok{alpha =} \FloatTok{0.5}\NormalTok{) }\SpecialCharTok{+}
  \FunctionTok{geom\_histogram}\NormalTok{(}\FunctionTok{aes}\NormalTok{(u), }\AttributeTok{bins =} \DecValTok{100}\NormalTok{, }\AttributeTok{color=}\StringTok{"white"}\NormalTok{, }\AttributeTok{fill=}\StringTok{"blue"}\NormalTok{, }\AttributeTok{alpha=}\FloatTok{0.5}\NormalTok{)}\SpecialCharTok{+}
  \FunctionTok{theme\_economist\_white}\NormalTok{() }\SpecialCharTok{+}
  \FunctionTok{ggtitle}\NormalTok{(}\StringTok{"Distribución de Variables sin correlacion"}\NormalTok{) }\SpecialCharTok{+}
  \FunctionTok{labs}\NormalTok{(}\AttributeTok{x =} \StringTok{"Valores de las Variables"}\NormalTok{, }\AttributeTok{y =} \StringTok{"Frecuencia"}\NormalTok{) }\SpecialCharTok{+} 
  \FunctionTok{theme}\NormalTok{(}
    \AttributeTok{plot.title =} \FunctionTok{element\_text}\NormalTok{(}\AttributeTok{face =} \StringTok{"bold"}\NormalTok{, }\AttributeTok{hjust =} \FloatTok{0.5}\NormalTok{),  }
    \AttributeTok{axis.title.x =} \FunctionTok{element\_text}\NormalTok{(}\AttributeTok{face =} \StringTok{"bold"}\NormalTok{),  }
    \AttributeTok{axis.title.y =} \FunctionTok{element\_text}\NormalTok{(}\AttributeTok{face =} \StringTok{"bold"}\NormalTok{)}
\NormalTok{  )}


\NormalTok{p2 }\OtherTok{\textless{}{-}} \FunctionTok{ggplot}\NormalTok{(df\_1) }\SpecialCharTok{+} 
  \FunctionTok{geom\_histogram}\NormalTok{(}\FunctionTok{aes}\NormalTok{(income), }\AttributeTok{bins =} \DecValTok{100}\NormalTok{, }\AttributeTok{color =} \StringTok{"white"}\NormalTok{, }\AttributeTok{fill =} \StringTok{"green"}\NormalTok{, }\AttributeTok{alpha =} \FloatTok{0.5}\NormalTok{) }\SpecialCharTok{+} 
  \FunctionTok{geom\_histogram}\NormalTok{(}\FunctionTok{aes}\NormalTok{(educ), }\AttributeTok{bins =} \DecValTok{100}\NormalTok{, }\AttributeTok{color =} \StringTok{"white"}\NormalTok{, }\AttributeTok{fill =} \StringTok{"orange"}\NormalTok{, }\AttributeTok{alpha =} \FloatTok{0.5}\NormalTok{) }\SpecialCharTok{+}
  \FunctionTok{geom\_histogram}\NormalTok{(}\FunctionTok{aes}\NormalTok{(u), }\AttributeTok{bins =} \DecValTok{100}\NormalTok{, }\AttributeTok{color=}\StringTok{"white"}\NormalTok{, }\AttributeTok{fill=}\StringTok{"blue"}\NormalTok{, }\AttributeTok{alpha=}\FloatTok{0.5}\NormalTok{)}\SpecialCharTok{+}
  \FunctionTok{theme\_economist\_white}\NormalTok{() }\SpecialCharTok{+}
  \FunctionTok{ggtitle}\NormalTok{(}\StringTok{"Distribución de Variables con correlacion"}\NormalTok{) }\SpecialCharTok{+}
  \FunctionTok{labs}\NormalTok{(}\AttributeTok{x =} \StringTok{"Valores de las Variables"}\NormalTok{, }\AttributeTok{y =} \StringTok{"Frecuencia"}\NormalTok{) }\SpecialCharTok{+} 
  \FunctionTok{theme}\NormalTok{(}
    \AttributeTok{plot.title =} \FunctionTok{element\_text}\NormalTok{(}\AttributeTok{face =} \StringTok{"bold"}\NormalTok{, }\AttributeTok{hjust =} \FloatTok{0.5}\NormalTok{),  }
    \AttributeTok{axis.title.x =} \FunctionTok{element\_text}\NormalTok{(}\AttributeTok{face =} \StringTok{"bold"}\NormalTok{),  }
    \AttributeTok{axis.title.y =} \FunctionTok{element\_text}\NormalTok{(}\AttributeTok{face =} \StringTok{"bold"}\NormalTok{)}
\NormalTok{  )}


\FunctionTok{grid.arrange}\NormalTok{(p1, p2, }\AttributeTok{ncol =} \DecValTok{2}\NormalTok{)}
\end{Highlighting}
\end{Shaded}

\includegraphics{Tarea_1_files/figure-latex/unnamed-chunk-2-1.pdf}

Dada la especificación del modelo, la distribución de las variables
sigue una distribución normal en ambos escenarios , es decir la
correlación entre las variable de tratamiento y el término de error
(para este ejemplo, la habilidad) en la segunda simulación no afecta la
distribución individual de la variable.

\begin{Shaded}
\begin{Highlighting}[]
\NormalTok{p1 }\OtherTok{\textless{}{-}} \FunctionTok{ggplot}\NormalTok{(df, }\FunctionTok{aes}\NormalTok{(}\AttributeTok{x =}\NormalTok{ u, }\AttributeTok{y =}\NormalTok{ niv\_educ, }\AttributeTok{group =}\NormalTok{ niv\_educ)) }\SpecialCharTok{+}
  \FunctionTok{geom\_boxplot}\NormalTok{(}\AttributeTok{alpha =} \FloatTok{0.3}\NormalTok{) }\SpecialCharTok{+}  
  \FunctionTok{coord\_flip}\NormalTok{() }\SpecialCharTok{+}
  \FunctionTok{theme\_economist\_white}\NormalTok{() }\SpecialCharTok{+}
  \FunctionTok{ggtitle}\NormalTok{(}\StringTok{"Distribución de u por Nivel Educativo (df)"}\NormalTok{) }\SpecialCharTok{+} 
  \FunctionTok{labs}\NormalTok{(}\AttributeTok{x =} \StringTok{"u"}\NormalTok{, }\AttributeTok{y =} \StringTok{"Nivel Educativo"}\NormalTok{) }\SpecialCharTok{+}
  \FunctionTok{theme}\NormalTok{(}
    \AttributeTok{plot.title =} \FunctionTok{element\_text}\NormalTok{(}\AttributeTok{face =} \StringTok{"bold"}\NormalTok{, }\AttributeTok{hjust =} \FloatTok{0.5}\NormalTok{),}
    \AttributeTok{axis.title.x =} \FunctionTok{element\_text}\NormalTok{(}\AttributeTok{face =} \StringTok{"bold"}\NormalTok{),}
    \AttributeTok{axis.title.y =} \FunctionTok{element\_text}\NormalTok{(}\AttributeTok{face =} \StringTok{"bold"}\NormalTok{)}
\NormalTok{  )}


\NormalTok{p2 }\OtherTok{\textless{}{-}} \FunctionTok{ggplot}\NormalTok{(df\_1, }\FunctionTok{aes}\NormalTok{(}\AttributeTok{x =}\NormalTok{ u, }\AttributeTok{y =}\NormalTok{ niv\_edu, }\AttributeTok{group =}\NormalTok{ niv\_edu)) }\SpecialCharTok{+}
  \FunctionTok{geom\_boxplot}\NormalTok{(}\AttributeTok{alpha =} \FloatTok{0.3}\NormalTok{) }\SpecialCharTok{+}  
  \FunctionTok{coord\_flip}\NormalTok{() }\SpecialCharTok{+}
  \FunctionTok{theme\_economist\_white}\NormalTok{() }\SpecialCharTok{+}
  \FunctionTok{ggtitle}\NormalTok{(}\StringTok{"Distribución de u por Nivel Educativo (df\_1)"}\NormalTok{) }\SpecialCharTok{+} 
  \FunctionTok{labs}\NormalTok{(}\AttributeTok{x =} \StringTok{"u"}\NormalTok{, }\AttributeTok{y =} \StringTok{"Nivel Educativo"}\NormalTok{) }\SpecialCharTok{+}
  \FunctionTok{theme}\NormalTok{(}
    \AttributeTok{plot.title =} \FunctionTok{element\_text}\NormalTok{(}\AttributeTok{face =} \StringTok{"bold"}\NormalTok{, }\AttributeTok{hjust =} \FloatTok{0.5}\NormalTok{),}
    \AttributeTok{axis.title.x =} \FunctionTok{element\_text}\NormalTok{(}\AttributeTok{face =} \StringTok{"bold"}\NormalTok{),}
    \AttributeTok{axis.title.y =} \FunctionTok{element\_text}\NormalTok{(}\AttributeTok{face =} \StringTok{"bold"}\NormalTok{)}
\NormalTok{  )}


\FunctionTok{grid.arrange}\NormalTok{(p1, p2, }\AttributeTok{ncol =} \DecValTok{2}\NormalTok{)}
\end{Highlighting}
\end{Shaded}

\includegraphics{Tarea_1_files/figure-latex/unnamed-chunk-3-1.pdf}

Para el ejercicio se definió una correlación entre la variable años de
educación y el termino de error de 0.9. Esto produce que el término de
error varie de forma incremental a mayor número de años de educación, es
decir, a mayor numero de años de educación se observa un incremento en
la media del término de error (correlación positiva) y por ende un mayor
ingreso económico, esto introduce un sesgo de selección que produce una
sobreestimación del efecto

\begin{Shaded}
\begin{Highlighting}[]
\NormalTok{p1 }\OtherTok{\textless{}{-}} \FunctionTok{ggplot}\NormalTok{(df, }\FunctionTok{aes}\NormalTok{(}\AttributeTok{x =}\NormalTok{ income, }\AttributeTok{y =}\NormalTok{ niv\_educ, }\AttributeTok{group =}\NormalTok{ niv\_educ)) }\SpecialCharTok{+}
  \FunctionTok{geom\_density\_ridges}\NormalTok{(}\AttributeTok{alpha =} \FloatTok{0.3}\NormalTok{) }\SpecialCharTok{+}
  \FunctionTok{geom\_function}\NormalTok{(}\AttributeTok{fun =} \ControlFlowTok{function}\NormalTok{(x) (}\SpecialCharTok{{-}}\NormalTok{income\_aut}\SpecialCharTok{/}\NormalTok{effect) }\SpecialCharTok{+}\NormalTok{ (}\DecValTok{1}\SpecialCharTok{/}\NormalTok{effect)}\SpecialCharTok{*}\NormalTok{x, }\AttributeTok{xlim =} \FunctionTok{c}\NormalTok{(}\DecValTok{100}\NormalTok{, }\DecValTok{400}\NormalTok{), }\AttributeTok{color =} \StringTok{"orange"}\NormalTok{) }\SpecialCharTok{+}
  \FunctionTok{coord\_flip}\NormalTok{() }\SpecialCharTok{+}
  \FunctionTok{theme\_economist\_white}\NormalTok{() }\SpecialCharTok{+}
  \FunctionTok{ggtitle}\NormalTok{(}\StringTok{"Distribución de Income por Nivel Educativo (df)"}\NormalTok{) }\SpecialCharTok{+} 
  \FunctionTok{labs}\NormalTok{(}\AttributeTok{x =} \StringTok{"Income"}\NormalTok{, }\AttributeTok{y =} \StringTok{"Nivel Educativo"}\NormalTok{) }\SpecialCharTok{+}
  \FunctionTok{theme}\NormalTok{(}
    \AttributeTok{plot.title =} \FunctionTok{element\_text}\NormalTok{(}\AttributeTok{face =} \StringTok{"bold"}\NormalTok{, }\AttributeTok{hjust =} \FloatTok{0.5}\NormalTok{),}
    \AttributeTok{axis.title.x =} \FunctionTok{element\_text}\NormalTok{(}\AttributeTok{face =} \StringTok{"bold"}\NormalTok{),}
    \AttributeTok{axis.title.y =} \FunctionTok{element\_text}\NormalTok{(}\AttributeTok{face =} \StringTok{"bold"}\NormalTok{)}
\NormalTok{  )}


\NormalTok{p2 }\OtherTok{\textless{}{-}} \FunctionTok{ggplot}\NormalTok{(df\_1, }\FunctionTok{aes}\NormalTok{(}\AttributeTok{x =}\NormalTok{ income, }\AttributeTok{y =}\NormalTok{ niv\_edu, }\AttributeTok{group =}\NormalTok{ niv\_edu)) }\SpecialCharTok{+}
  \FunctionTok{geom\_density\_ridges}\NormalTok{(}\AttributeTok{alpha =} \FloatTok{0.3}\NormalTok{) }\SpecialCharTok{+}
  \FunctionTok{geom\_function}\NormalTok{(}\AttributeTok{fun =} \ControlFlowTok{function}\NormalTok{(x) (}\SpecialCharTok{{-}}\NormalTok{income\_aut}\SpecialCharTok{/}\NormalTok{effect) }\SpecialCharTok{+}\NormalTok{ (}\DecValTok{1}\SpecialCharTok{/}\NormalTok{effect)}\SpecialCharTok{*}\NormalTok{x, }\AttributeTok{xlim =} \FunctionTok{c}\NormalTok{(}\DecValTok{100}\NormalTok{, }\DecValTok{400}\NormalTok{), }\AttributeTok{color =} \StringTok{"orange"}\NormalTok{) }\SpecialCharTok{+}
  \FunctionTok{coord\_flip}\NormalTok{() }\SpecialCharTok{+}
  \FunctionTok{theme\_economist\_white}\NormalTok{() }\SpecialCharTok{+}
  \FunctionTok{ggtitle}\NormalTok{(}\StringTok{"Distribución de Income por Nivel Educativo (df\_1)"}\NormalTok{) }\SpecialCharTok{+} 
  \FunctionTok{labs}\NormalTok{(}\AttributeTok{x =} \StringTok{"Income"}\NormalTok{, }\AttributeTok{y =} \StringTok{"Nivel Educativo"}\NormalTok{) }\SpecialCharTok{+}
  \FunctionTok{theme}\NormalTok{(}
    \AttributeTok{plot.title =} \FunctionTok{element\_text}\NormalTok{(}\AttributeTok{face =} \StringTok{"bold"}\NormalTok{, }\AttributeTok{hjust =} \FloatTok{0.5}\NormalTok{),}
    \AttributeTok{axis.title.x =} \FunctionTok{element\_text}\NormalTok{(}\AttributeTok{face =} \StringTok{"bold"}\NormalTok{),}
    \AttributeTok{axis.title.y =} \FunctionTok{element\_text}\NormalTok{(}\AttributeTok{face =} \StringTok{"bold"}\NormalTok{)}
\NormalTok{  )}


\FunctionTok{grid.arrange}\NormalTok{(p1, p2, }\AttributeTok{ncol =} \DecValTok{2}\NormalTok{)}
\end{Highlighting}
\end{Shaded}

\includegraphics{Tarea_1_files/figure-latex/unnamed-chunk-4-1.pdf}

En ambos casos se observa una correlación positiva entre el nivel
educativo y el ingreso. No obstante, en la simulación 2, se tiene un
efecto mayor (una pendiente más pronunciada) entre el nivel educativo y
el ingreso, esto es secundario al sesgo de selección que sobreestima el
efecto del nivel educativo.

\begin{Shaded}
\begin{Highlighting}[]
\NormalTok{media\_df }\OtherTok{\textless{}{-}} \FunctionTok{mean}\NormalTok{(reg}\SpecialCharTok{$}\NormalTok{estimate, }\AttributeTok{na.rm =} \ConstantTok{TRUE}\NormalTok{)}
\NormalTok{media\_df1 }\OtherTok{\textless{}{-}} \FunctionTok{mean}\NormalTok{(reg\_1}\SpecialCharTok{$}\NormalTok{estimate, }\AttributeTok{na.rm =} \ConstantTok{TRUE}\NormalTok{)}


\NormalTok{p1 }\OtherTok{\textless{}{-}} \FunctionTok{ggplot}\NormalTok{(reg, }\FunctionTok{aes}\NormalTok{(estimate)) }\SpecialCharTok{+} 
  \FunctionTok{geom\_histogram}\NormalTok{(}\AttributeTok{color =} \StringTok{"white"}\NormalTok{, }\AttributeTok{bins =} \DecValTok{100}\NormalTok{, }\AttributeTok{alpha =} \FloatTok{0.5}\NormalTok{) }\SpecialCharTok{+} 
  \FunctionTok{geom\_vline}\NormalTok{(}\FunctionTok{aes}\NormalTok{(}\AttributeTok{xintercept =}\NormalTok{ media\_df), }\AttributeTok{color =} \StringTok{"red"}\NormalTok{, }\AttributeTok{linetype =} \StringTok{"dashed"}\NormalTok{) }\SpecialCharTok{+}
  \FunctionTok{theme\_economist\_white}\NormalTok{() }\SpecialCharTok{+}
  \FunctionTok{ggtitle}\NormalTok{(}\StringTok{"Distribución de Estimates (reg)"}\NormalTok{) }\SpecialCharTok{+}
  \FunctionTok{theme}\NormalTok{(}\AttributeTok{plot.title =} \FunctionTok{element\_text}\NormalTok{(}\AttributeTok{face =} \StringTok{"bold"}\NormalTok{, }\AttributeTok{hjust =} \FloatTok{0.5}\NormalTok{))}


\NormalTok{p2 }\OtherTok{\textless{}{-}} \FunctionTok{ggplot}\NormalTok{(reg\_1, }\FunctionTok{aes}\NormalTok{(estimate)) }\SpecialCharTok{+} 
  \FunctionTok{geom\_histogram}\NormalTok{(}\AttributeTok{color =} \StringTok{"white"}\NormalTok{, }\AttributeTok{bins =} \DecValTok{100}\NormalTok{, }\AttributeTok{alpha =} \FloatTok{0.5}\NormalTok{) }\SpecialCharTok{+} 
  \FunctionTok{geom\_vline}\NormalTok{(}\FunctionTok{aes}\NormalTok{(}\AttributeTok{xintercept =}\NormalTok{ media\_df1), }\AttributeTok{color =} \StringTok{"red"}\NormalTok{, }\AttributeTok{linetype =} \StringTok{"dashed"}\NormalTok{) }\SpecialCharTok{+}
  \FunctionTok{theme\_economist\_white}\NormalTok{() }\SpecialCharTok{+}
  \FunctionTok{ggtitle}\NormalTok{(}\StringTok{"Distribución de Estimates (df1)"}\NormalTok{) }\SpecialCharTok{+}
  \FunctionTok{theme}\NormalTok{(}\AttributeTok{plot.title =} \FunctionTok{element\_text}\NormalTok{(}\AttributeTok{face =} \StringTok{"bold"}\NormalTok{, }\AttributeTok{hjust =} \FloatTok{0.5}\NormalTok{))}



\FunctionTok{grid.arrange}\NormalTok{(p1, p2, }\AttributeTok{ncol =} \DecValTok{2}\NormalTok{)}
\end{Highlighting}
\end{Shaded}

\includegraphics{Tarea_1_files/figure-latex/unnamed-chunk-5-1.pdf}

\begin{Shaded}
\begin{Highlighting}[]
\NormalTok{df\_h0 }\OtherTok{\textless{}{-}}\NormalTok{ reg }\SpecialCharTok{\%\textgreater{}\%} 
         \FunctionTok{select}\NormalTok{(estimate, conf.low, conf.high, reg)  }\SpecialCharTok{\%\textgreater{}\%}
         \FunctionTok{mutate}\NormalTok{(}\AttributeTok{h0=}\DecValTok{10}\NormalTok{,}
               \AttributeTok{rechaza\_h0=}\FunctionTok{ifelse}\NormalTok{(conf.low }\SpecialCharTok{\textless{}}\NormalTok{ h0 }\SpecialCharTok{\&}\NormalTok{  h0 }\SpecialCharTok{\textless{}}\NormalTok{ conf.high , }\StringTok{"No"}\NormalTok{, }\StringTok{"Si"}\NormalTok{)) }\SpecialCharTok{\%\textgreater{}\%}
             \FunctionTok{arrange}\NormalTok{(estimate) }\SpecialCharTok{\%\textgreater{}\%}
             \FunctionTok{mutate}\NormalTok{(}\AttributeTok{reg=}\FunctionTok{c}\NormalTok{(}\DecValTok{1}\SpecialCharTok{:}\FunctionTok{nrow}\NormalTok{(.)))}

\NormalTok{df\_h0\_1 }\OtherTok{\textless{}{-}}\NormalTok{ reg\_1 }\SpecialCharTok{\%\textgreater{}\%} 
  \FunctionTok{select}\NormalTok{(estimate, conf.low, conf.high, reg)  }\SpecialCharTok{\%\textgreater{}\%}
  \FunctionTok{mutate}\NormalTok{(}\AttributeTok{h0 =} \DecValTok{10}\NormalTok{,}
         \AttributeTok{rechaza\_h0 =} \FunctionTok{ifelse}\NormalTok{(conf.low }\SpecialCharTok{\textless{}}\NormalTok{ h0 }\SpecialCharTok{\&}\NormalTok{ h0 }\SpecialCharTok{\textless{}}\NormalTok{ conf.high, }\StringTok{"No"}\NormalTok{, }\StringTok{"Si"}\NormalTok{)) }\SpecialCharTok{\%\textgreater{}\%}
  \FunctionTok{arrange}\NormalTok{(estimate) }\SpecialCharTok{\%\textgreater{}\%}
  \FunctionTok{mutate}\NormalTok{(}\AttributeTok{reg =} \FunctionTok{c}\NormalTok{(}\DecValTok{1}\SpecialCharTok{:}\FunctionTok{nrow}\NormalTok{(.)))}


\NormalTok{p1 }\OtherTok{\textless{}{-}} \FunctionTok{ggplot}\NormalTok{(df\_h0, }\FunctionTok{aes}\NormalTok{(}\AttributeTok{x =}\NormalTok{ reg)) }\SpecialCharTok{+}
  \FunctionTok{geom\_linerange}\NormalTok{(}\FunctionTok{aes}\NormalTok{(}\AttributeTok{ymin =}\NormalTok{ conf.low, }\AttributeTok{ymax =}\NormalTok{ conf.high, }\AttributeTok{color =}\NormalTok{ rechaza\_h0), }\AttributeTok{lwd =} \FloatTok{0.1}\NormalTok{) }\SpecialCharTok{+} 
  \FunctionTok{scale\_color\_manual}\NormalTok{(}\AttributeTok{values =} \FunctionTok{c}\NormalTok{(}\StringTok{"No"}\OtherTok{=} \StringTok{"\#999999"}\NormalTok{,}\StringTok{"Si"} \OtherTok{=} \StringTok{"\#ef8a62"}\NormalTok{)) }\SpecialCharTok{+}
  \FunctionTok{geom\_point}\NormalTok{(}\FunctionTok{aes}\NormalTok{(}\AttributeTok{y =}\NormalTok{ estimate), }\AttributeTok{size =} \FloatTok{0.1}\NormalTok{) }\SpecialCharTok{+} 
  \FunctionTok{geom\_hline}\NormalTok{(}\FunctionTok{aes}\NormalTok{(}\AttributeTok{yintercept =}\NormalTok{ h0), }\AttributeTok{color =} \StringTok{"red"}\NormalTok{, }\AttributeTok{linetype =} \StringTok{"dashed"}\NormalTok{) }\SpecialCharTok{+}
  \FunctionTok{labs}\NormalTok{(}\AttributeTok{y =} \StringTok{"coef"}\NormalTok{, }\AttributeTok{x =} \StringTok{"reg"}\NormalTok{) }\SpecialCharTok{+}
  \FunctionTok{theme\_economist\_white}\NormalTok{() }\SpecialCharTok{+}
  \FunctionTok{ggtitle}\NormalTok{(}\StringTok{"Distribución de Coeficientes (reg)"}\NormalTok{) }\SpecialCharTok{+}
  \FunctionTok{theme}\NormalTok{(}\AttributeTok{plot.title =} \FunctionTok{element\_text}\NormalTok{(}\AttributeTok{face =} \StringTok{"bold"}\NormalTok{, }\AttributeTok{hjust =} \FloatTok{0.5}\NormalTok{))}


\NormalTok{p2 }\OtherTok{\textless{}{-}} \FunctionTok{ggplot}\NormalTok{(df\_h0\_1, }\FunctionTok{aes}\NormalTok{(}\AttributeTok{x =}\NormalTok{ reg)) }\SpecialCharTok{+}
  \FunctionTok{geom\_linerange}\NormalTok{(}\FunctionTok{aes}\NormalTok{(}\AttributeTok{ymin =}\NormalTok{ conf.low, }\AttributeTok{ymax =}\NormalTok{ conf.high, }\AttributeTok{color =}\NormalTok{ rechaza\_h0), }\AttributeTok{lwd =} \FloatTok{0.1}\NormalTok{) }\SpecialCharTok{+} 
  \FunctionTok{scale\_color\_manual}\NormalTok{(}\AttributeTok{values =} \FunctionTok{c}\NormalTok{(}\StringTok{"No"}\OtherTok{=} \StringTok{"\#999999"}\NormalTok{,}\StringTok{"Si"} \OtherTok{=} \StringTok{"\#ef8a62"}\NormalTok{)) }\SpecialCharTok{+}
  \FunctionTok{geom\_point}\NormalTok{(}\FunctionTok{aes}\NormalTok{(}\AttributeTok{y =}\NormalTok{ estimate), }\AttributeTok{size =} \FloatTok{0.1}\NormalTok{) }\SpecialCharTok{+} 
  \FunctionTok{geom\_hline}\NormalTok{(}\FunctionTok{aes}\NormalTok{(}\AttributeTok{yintercept =}\NormalTok{ h0), }\AttributeTok{color =} \StringTok{"red"}\NormalTok{, }\AttributeTok{linetype =} \StringTok{"dashed"}\NormalTok{) }\SpecialCharTok{+}
  \FunctionTok{labs}\NormalTok{(}\AttributeTok{y =} \StringTok{"coef"}\NormalTok{, }\AttributeTok{x =} \StringTok{"reg"}\NormalTok{) }\SpecialCharTok{+}
  \FunctionTok{theme\_economist\_white}\NormalTok{() }\SpecialCharTok{+}
  \FunctionTok{ggtitle}\NormalTok{(}\StringTok{"Distribución de Coeficientes (reg\_1)"}\NormalTok{) }\SpecialCharTok{+}
  \FunctionTok{theme}\NormalTok{(}\AttributeTok{plot.title =} \FunctionTok{element\_text}\NormalTok{(}\AttributeTok{face =} \StringTok{"bold"}\NormalTok{, }\AttributeTok{hjust =} \FloatTok{0.5}\NormalTok{))}



\FunctionTok{grid.arrange}\NormalTok{(p1, p2, }\AttributeTok{ncol =} \DecValTok{2}\NormalTok{)}
\end{Highlighting}
\end{Shaded}

\includegraphics{Tarea_1_files/figure-latex/unnamed-chunk-6-1.pdf}

Se observa que cuando se cumple el supuesto de exogeneidad, el parámetro
poblacional (\(\alpha\)) que representa el efecto de la educación sobre
los ingresos se encuentra dentro del intervalo de confianza en el 95\%
de las estimaciones. Por otra parte, cuando violamos este supuesto en
ningún caso se logra una estimación en la que el verdadero parámetro
poblacional esté dentro del intervalo de confianza.

\section{Conclusion}\label{conclusion}

El objetivo de este ejercicio es evaluar los supuestos básicos del
modelo de regresión lineal en situaciones en las que los datos no
cumplen con las condiciones ideales del estimador. Inicialmente, se
realizan estimaciones utilizando un modelo de regresión lineal en un
contexto donde los datos son generados aleatoriamente, asumiendo que no
existe correlación entre el término de error (habilidad) y el nivel de
educación.

Esto proporciona una referencia para evaluar el rendimiento del
estimador bajo condiciones ideales. En una segunda fase, se introduce
una correlación entre la educación y el término de error, reflejando una
situación más realista donde el nivel de habilidad no observada puede
influir tanto en el nivel educativo como en el ingreso. Esta correlación
puede sesgar el estimador del impacto de la educación sobre el ingreso,
ya que el término de error, que captura factores no observables que
afectan el ingreso, también está relacionado con el nivel educativo.

Como resultado, la estimación del efecto de la educación puede no
reflejar con precisión la verdadera relación entre educación e ingreso.
Este ejercicio subraya la importancia de verificar los supuestos básicos
del modelo de regresión lineal, como la independencia entre el término
de error y las variables explicativas, ya que la violación de estos
supuestos puede llevar a estimaciones sesgadas e incorrectas, lo que
resalta la necesidad de técnicas adicionales para obtener estimaciones
más confiables en presencia de sesgo.

\end{document}
